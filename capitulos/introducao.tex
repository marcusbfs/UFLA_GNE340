\chapter{Introdução} \label{sec:intro}

\section{Referencial Teórico}\label{sec:refteo}

Reaproveitar materiais e fazer de seu novo uso algo útil, não só é tendência,
como algo necessário, pois com o avanço tecnológico e crescimento populacional,
a preocupação com es recursos naturais tende a aumentar. Reutilizar recipientes
metálicos para construir um filtro, além de evitar deposição de materiais no
meio ambiente, como poluentes, pode ser uma alternativa de baixo custo para
tratamento de água, antes de despejá-la em efluentes, unindo desenvolvimento de
tecnologia de baixo custo e funcionalidade com preservação do meio ambiente.

Um dos procedimentos mais simples de separação de sólidos e líquidos é a
filtração, aplicada em muitas etapas de processos da indústria química. Entra no
filtro a mistura a ser separada e como produtos, saem o filtrado (líquido
clarificado) e a torta de filtragem (sólido com um pouco de líquido). Ao longo
do processo, a própria torta se torna o meio filtrante, e como no início ela
está sendo formada, para que seja efetiva a filtração, os volumes iniciais
retornam ao filtro. Durante a filtração, características como altura,
permeabilidade e porosidade da torta, variam, também há variação de pressão ao
longo do processo, alterando a vazão. As variáveis devem ser controladas para
cálculo da velocidade da filtração e consequentemente o tempo gasto, sempre
buscando otimizar o processo \citeayp{isenmann2012operaccoes} .

\subsection{Tipos de filtro}

\label{subsec:tiposdefiltro}

\begin{enumerate}

\item[-] Filtro de pressão
 
  São filtros que funcionam em batelada, ou de forma contínua, operam
  pressurizados e costumam ter uma ou mais camadas de material granular. São
  usados geralmente em estações de tratamento de água. O filtro prensa é deste
  tipo, tem como vantagens a necessidade de uma menor área para sua implantação,
  São produzidos líquidos límpidos por meio da circulação do filtrado, as tortas
  resultantes apresentam baixa umidade e pode ser automatizado. Porém, tem as
  desvantagens de difícil lavagem e manutenção, as placas podem sofrer fissuras
  e romper-se e a técnica é muito sensível às variações das características dos
  resíduos.


\item[-] Filtro a vácuo

  Os filtros a vácuo podem ser alimentados no fundo ou no topo do equipamento.
  Caracteriza-se por conduzir tortas secas de pequena espessura, e operar
  continuamente sob baixa queda de pressão. É utilizado na indústria
  sucroalcooleira, no modelo contínuo de tambor rotativo a vácuo, por exemplo.
  Suas tortas apresentam maior quantidade de umidade residual se comparado ao
  filtro prensa, o meio filtrante requer lavagem constante e consome bastante
  energia, mas em compensação, a unidade precisa de pequena área de implantação,
  a torta é facilmente removível, tem fácil controle operacional e manutenção de
  baixo custo.


\end{enumerate}


\subsection{Meios filtrantes}
\label{subsec:meiosfiltrantes}

Para escolher um meio filtrante, precisam ser analisadas anteriormente
características que serão essenciais na otimização da filtração. É importante
que ao entrar em contato com a suspensão que será tratada, o meio filtrante não
sofra fissuras, rompimentos ou ataques químicos. Ter boa e adequada distribuição
dos poros faz com que o curso da filtração não seja comprometido, seja fácil
fazer a limpeza e o custo seja baixo.

Após o início da operação, a torta se tornará meio filtrante, por isso para
obter filtrado límpido, é importante voltar os primeiros volumes ao processo,
visto que nas primeiras porções de filtrado há traços de particulado.

Alguns meios utilizados que merecem ser citados são algodão, polímeros
sintéticos, metais, e no caso de filtros granulares, cascalho, areia, antracito
e carvão ativado.


\section{Aplicabilidade}\label{sec:aplicabilidade}

Gabi

\section{Objetivo}\label{sec:objetivos} 

O projeto terá como objetivo a montagem de um filtro prensa a partir de
materiais acessíveis, tal como sua operação na filtração de um fluido contendo
partículas indesejáveis.


%%% Local Variables:
%%% mode: latex
%%% TeX-master: "../main_archive"
%%% End:
