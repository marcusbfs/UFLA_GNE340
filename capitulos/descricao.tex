\chapter{Descrição do Projeto}\label{descricao}

\section{Materiais Utilizados}

\begin{table}[H]
\centering
\caption{Materiais utilizados no projeto}
\label{tab:materiais}
\begin{tabular}{lc}
\textbf{Materiais}         & \multicolumn{1}{l}{\textbf{Custos (R\$)}} \\ \toprule
Algodão                    &      14,00 pacote                   \\
Bomba de ar para bicicleta &      19,00 cada                     \\
Garrafa PET                & -                                   \\
Graxa                      &       7,00 cada                     \\
Latas redondas             & -                                   \\
Mangueira                  &       4,00 cada                     \\
Parafuso de rosca sem fim  &       2,00 metro                    \\
Placas de madeira          & -                                   \\
  Porcas                   &       8,00 pacote                   \\ \bottomrule
\end{tabular}
\end{table}

\section{Montagem}
\label{sec:montagem}

\begin{itemize}
\item Realizar quatro furos nas bordas das placas de madeira ($30\times30$ \si{cm}),
  para que seja possível acoplar os quatro parafusos que irão unir o sistema;
\item Em uma das placas de madeira realizar um furo central com aproximadamente
  $1/4$ \si{in}, que servirá como encaixe para a mangueira responsável pelo canal de
  alimentação do filtro;
\item As membranas filtrantes serão formadas por latas preenchidas com algodão e
  vedadas com graxa, para que não ocorra nenhum tipo de vazamento. Um furo
  central deve ser realizado em cada membrana, para a passagem do fluido;
\item Uma mangueira irá conectar o protótipo ao recipiente que armazenará o
  fluido, sendo que no centro deste canal de escoamento deverá ser conectado
  uma bomba de ar, que terá a função de pressurizar o sistema impedindo que o
  fluido tenha um escoamento reverso, que não é desejado.
\end{itemize}

Os testes serão realizados fazendo com que o fluido contaminado armazenado em um
recipiente localizado em um nível acima do filtro tipo prensa (para que seja
possível o escoamento por gravidade), escoe pelo filtro montado e que o fluido
obtido deste precesso seja um fluido descontaminado.

%%% Local Variables:
%%% mode: latex
%%% TeX-master: "../main_archive"
%%% End:
