\chapter{Descrição do Projeto}\label{descricao}

\section{Materiais Utilizados}

\begin{table}[H]
\centering
\caption{Materiais utilizados no projeto}
\label{tab:materiais}
\begin{tabular}{lc}
\textbf{Materiais}         & \multicolumn{1}{l}{\textbf{Custos (R\$)}} \\ \toprule
Algodão                    &      14,00 pacote                   \\
Bomba de ar para bicicleta &      19,00 cada                     \\
Garrafa PET                & -                                   \\
Graxa                      &       7,00 cada                     \\
Latas redondas             & -                                   \\
Mangueira                  &       4,00 cada                     \\
Parafuso de rosca sem fim  &       2,00 metro                    \\
Placas de madeira          & -                                   \\
  Porcas                   &       8,00 pacote                   \\ \bottomrule
\end{tabular}
\end{table}

\section{Montagem}
\label{sec:montagem}

Primeiramente, foram realizados quatro furos nos cantos das placas de madeira,
que é onde os parafusos seriam acoplados para unir o sistema. Em uma das placas
de madeira, foi feito um furo central de \nicefrac{1}{4} \si{in}, para encaixe
da mangueira de alimentação do filtro. Na montagem da membrana filtrante (latas
preenchidas com algodão), foram feitos furos de \nicefrac{1}{4} \si{in} no
centro de cada lata, por onde a solução a ser filtrada entraria no meio
filtrante (algodão).

Foram conectadas as duas placas por meio dos parafusos de rosca sem fim, de modo
a deixar um espaço para colocar as membranas filtrantes. As latas foram
preenchidas com algodão e tampadas, empilhando e encaixando-as entre as duas
placas de madeira, de modo que as membranas filtrantes ficaram prensadas entre
as placas de madeira. Uma mangueira foi conectada ao furo central da placa de
madeira, por onde seria feita a alimentação. Foi utilizada uma garrafa PET para
colocar a solução a ser filtrada.



%%% Local Variables:
%%% mode: latex
%%% TeX-master: "../main_archive"
%%% End:
